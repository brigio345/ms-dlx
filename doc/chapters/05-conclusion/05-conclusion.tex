\chapter{Conclusion}
\label{chap:conclusion}
\paragraph{Philosophy} \mbox{} \\
The DLXY was designed with \textbf{simplicity} in mind, both in architecture
choices and in source code management, but it is very difficult to tackle all
the issues without growing in complexity.

\paragraph{Performance}	\mbox{} \\
The DLXY performance is quite decent: it can run at a clock frequency around 1 GHz,
requiring less than 7 mW of total power.

\bigskip
At the steady state the CPI is 1 but it can easily arise toward 2 due to the
frequent presence of taken branches, each one requiring 2 clock cycles to be executed.
An improvement could be achieved anticipating the unconditional branches detection
in the fetch stage, in order to get them executed in 1 clock cycle, instead of 2.

\paragraph{Further improvements} \mbox{} \\
A list of possible additions or modifications to the DLXY may include:
\begin{itemize}
	\item advanced branch prediction techniques
	\item interrupts and exceptions handling
	\item division (introducing multi-cycle operations)
	\item caches hierarchy
\end{itemize}

