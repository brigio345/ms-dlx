\chapter{Synthesis}
\label{chap:synthesis}

\lstset{
	basicstyle=\tiny,
	frame=single,
	breaklines=true
}

\section{Scripts}
Script \texttt{src/script/synthesis.tcl} contains a set of \texttt{TCL}
procedures useful for synthesis automation, compatible with
\textit{DC Ultra - Version O-2018.06-SP4}.

\paragraph{\texttt{analyze\_dir}}
\begin{itemize}
	\item \underline{Argument(s):}
		\begin{itemize}
			\item \texttt{dir}: name of the directory containing
				source code to be compiled
			\item \texttt{src\_format}: format of source files
				to be analyzed (default is \textit{vhdl})
		\end{itemize}
	\item \underline{Result:} \\
		recursively analyzes all the entities contained in \texttt{dir},
		using the \texttt{analyze} command. \\
		Analysis starts from \textbf{deepest directory upwards} (in
		\textbf{reverse alphanumerical order}): this, toghether with the
		files organization described in appendix
		\ref{app:code_conventions} avoids any dependency analysis errors.
\end{itemize}

\paragraph{\texttt{synthesize}}
\begin{itemize}
	\item \underline{Argument(s):}
		\begin{itemize}
			\item \texttt{top\_entity}: name of top level entity to
				be synthesized (default is \texttt{dlx})
			\item \texttt{time\_contraint}: maximum delay between
				inputs and output and clock period in ns
				(default is \texttt{1.0005})
			\item \texttt{lvt\_lib}: low threshold voltage library
				(default is \texttt{CORE65LPLVT})
			\item \texttt{hvt\_lib}: high threshold voltage library
				(default is \texttt{CORE65LPHVT})
		\end{itemize}
	\item \underline{Result:} \\
		Run synthesis using \texttt{compile\_ultra} command, with
		\textbf{clock gating} enabled.
		\begin{itemize}
			\item \texttt{top\_entity-YYMMDDhhmm-postsyn.v}: verilog
				file containing the post-sythesis netlist
			\item \texttt{top\_entity-yyMMDDhhmm.sdc}
			\item \texttt{top\_entity-YYMMDDhhmm-timing.rpt}: timing
				report
			\item \texttt{top\_entity-YYMMDDhhmm-power.rpt}: power
				report
			\item \texttt{top\_entity-YYMMDDhhmm-area.rpt}: area
				report
			\item \texttt{top\_entity-YYMMDDhhmm-threshold.rpt}:
				thresold voltage group report
			\item \texttt{top\_entity-YYMMDDhhmm-gating.rpt}: clock
				gating report
		\end{itemize}
\end{itemize}

\paragraph{\texttt{synthesize\_dual\_vth}}
\begin{itemize}
	\item \underline{Argument(s):}
		\begin{itemize}
			\item \texttt{top\_entity}: name of top level entity to
				be synthesized (default is \texttt{dlx})
			\item \texttt{time\_contraint}: maximum delay between
				inputs and output and clock period in ns
				(default is \texttt{1.0005})
			\item \texttt{wire\_model}: (default is \texttt{5K\_hvratio\_1\_4})
		\end{itemize}
	\item \underline{Result:} \\
		Run synthesis using \texttt{compile\_ultra} command, with
		\textbf{dual V$_{\text{th}}$} and \textbf{clock gating} enabled.
		\begin{itemize}
			\item \texttt{top\_entity-YYMMDDhhmm-postsyn.v}: verilog
				file containing the post-sythesis netlist
			\item \texttt{top\_entity-yyMMDDhhmm.sdc}
			\item \texttt{top\_entity-YYMMDDhhmm-timing.rpt}: timing
				report
			\item \texttt{top\_entity-YYMMDDhhmm-power.rpt}: power
				report
			\item \texttt{top\_entity-YYMMDDhhmm-area.rpt}: area
				report
			\item \texttt{top\_entity-YYMMDDhhmm-gating.rpt}: clock
				gating report
		\end{itemize}
\end{itemize}

\section{Reports}
\subsection{Standard synthesis}
\label{subsec:std_syn}
The following reports have been obtained running
\begin{enumerate}
	\item \texttt{analyze\_dir src/main}
	\item \texttt{synthesize dlx 1.5}
\end{enumerate}

\bigskip
This way the DLXY synthesis has been performed with the following setup:.
\begin{itemize}
	\item \underline{Library}: \textit{NandgateOpenCellLibrary}
	\item \underline{Clock period and maximum delay:} \texttt{1.5 ns}
	\item \underline{Clock gating}
\end{itemize}

\paragraph{Timing report} \mbox{} \\
\lstinputlisting{chapters/03-synthesis/files/timing-std.rpt}
\paragraph{Power report} \mbox{} \\
\lstinputlisting{chapters/03-synthesis/files/power-std.rpt}
\paragraph{Area report} \mbox{} \\
\lstinputlisting{chapters/03-synthesis/files/area-std.rpt}
\paragraph{Clock gating report} \mbox{} \\
\lstinputlisting{chapters/03-synthesis/files/gating-std.rpt}

\subsection{Dual V-th synthesis}
The following reports have been obtained running
\begin{enumerate}
	\item \texttt{analyze\_dir src/main}
	\item \texttt{synthesize\_dual\_vth}
\end{enumerate}

\bigskip
This way the DLXY synthesis has been performed with the following setup:.
\begin{itemize}
	\item \underline{Library}: \textit{STcmos65}
	\item \underline{Clock period and maximum delay:} \texttt{1.0005 ns}
	\item \underline{Dual V-th}
	\item \underline{Clock gating}
\end{itemize}

\paragraph{Timing report} \mbox{} \\
\lstinputlisting{chapters/03-synthesis/files/timing-dual_vth.rpt}
\paragraph{Power report} \mbox{} \\
\lstinputlisting{chapters/03-synthesis/files/power-dual_vth.rpt}
\paragraph{Area report} \mbox{} \\
\lstinputlisting{chapters/03-synthesis/files/area-dual_vth.rpt}
\paragraph{Clock gating report} \mbox{} \\
\lstinputlisting{chapters/03-synthesis/files/gating-dual_vth.rpt}
\paragraph{Threshold voltage group report} \mbox{} \\
\lstinputlisting{chapters/03-synthesis/files/threshold-dual_vth.rpt}

