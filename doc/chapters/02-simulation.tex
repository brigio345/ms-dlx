\chapter{Simulation}
\label{chap:simulation}

\section{Approach}
The DLXY code is written and tested according to a \textbf{bottom-up} approach:
\begin{enumerate}
	\item test \textbf{low level entities} (i.e. not containing other
		entities as components)
	\item \label{it:test} test higher level \textbf{entities containing
		tested components} only
	\item iterate \ref{it:test} until top level entity is reached
\end{enumerate}

Testbenches make use of the \texttt{assert} command whenever possible, since
it makes very easy to detect possible errors, without having to manually scan
the waveforms on the GUI.

\bigskip
The top level testbench includes not only the \textbf{DLXY entity}, but also the
\textbf{memories}. The instruction memory is loaded from a \texttt{mem} file,
generated by the assembler located in \texttt{script/assembly}, starting from
a compatible \texttt{asm} file.

\section{Scripts}
Script \texttt{src/script/simulation.tcl} contains a set of \texttt{TCL}
procedures useful for simulation automation, compatible with
\textit{ModelSim SE-64 6.5c}.
\subsection{\texttt{compile\_all}}
The \texttt{compile\_all} procedure recursively compiles all the entities
contained in the directory specified as parameter, using the \texttt{vcom} command.

The compilation starts from deepest directory upwards: this, toghether with the
files organization described in appendix \ref{app:code_conventions} avoids any
dependency compilation errors.

