\chapter{Simulation}
\label{chap:simulation}

\section{Approach}
The DLXY code has been tested according to a \textbf{bottom-up} approach:
\begin{enumerate}
	\item test \textbf{low level entities} (i.e. not containing other
		entities as components)
	\item \label{it:test} test higher level \textbf{entities containing
		tested components} only
	\item iterate \ref{it:test} until top level entity is reached
\end{enumerate}

Testbenches make use of the \texttt{assert} command whenever possible, since
it makes very easy to detect possible errors, without having to manually scan
the waveforms on the GUI.

\bigskip
The top level testbench includes not only the \textbf{DLXY entity}, but also the
\textbf{memories}. The instruction memory is loaded from a \texttt{mem} file,
generated by the assembler located in \texttt{script/assembly}, starting from
a compatible \texttt{asm} file.

\section{Scripts}
Script \texttt{src/script/simulation.tcl} contains a set of \texttt{TCL}
procedures useful for simulation automation, compatible with
\textit{ModelSim SE-64 6.5c}.
\subsection{\texttt{compile\_dir}}
\begin{itemize}
	\item \underline{Argument(s):}
		\begin{itemize}
			\item \texttt{dir}: name of the directory containing
				source code to be compiled
			\item \texttt{file\_extension}: extension of source files
				to be compiled (default is \textit{vhd})
		\end{itemize}
	\item \underline{Result:} \\
		recursively compiles all the entities contained in \texttt{dir},
		using the \texttt{vcom} command. \\
		Compilation starts from \textbf{deepest directory upwards} (in
		\textbf{reverse alphanumerical order}): this, toghether with the
		files organization described in appendix
		\ref{app:code_conventions} avoids any dependency compilation errors.
\end{itemize}

\subsection{\texttt{simulate\_dlx}}
\begin{itemize}
	\item \underline{Argument(s):}
		\begin{itemize}
			\item \texttt{asm\_file}: assembly source code to be
				loaded to instruction memory
			\item \texttt{run\_time}: duration (in ns) of the
				simulation
		\end{itemize}
	\item \underline{Result:} \\
		make the DLXY run \texttt{asm\_file} code for \texttt{run\_time}
		nanoseconds (or 100 ns if \texttt{run\_time} is not specified)
\end{itemize}

